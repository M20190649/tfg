\documentclass[../proyecto.tex]{subfiles}

\begin{document}
\chapter{Estado del arte}

Tradicionalmente el conteo de vehículos en vías públicas se ha realizado mediante detectores desplegados \textit{in situ}, estos detectores se despliegan en un emplazamiento fijo donde se desea realizar la medición, normalmente en sitios de especial interés como intersecciones importantes. Estas tecnologías se podrían diferenciar en dos grupos: intrusivos y no intrusivos. Los intrusivos se despliegan en la propia vía ya sea sobre ella o bajo el pavimento, por tanto su instalación y mantenimiento suelen requerir interrumpir el tráfico, algunos ejemplos de este tipo de sistemas de detección son: tubos neumáticos, detectores de bucle de inducción, sensores magnéticos y sensores piezoeléctricos. Por otra parte los detectores no intrusivos se despliegan a los lados de la vía o sobre ella utilizando estructuras existente como puentes o señales de tráfico, como ejemplos tenemos: detectores infrarrojos, radares, sensores de ultrasonidos y  procesamiento de imágenes usando cámaras de circuito cerrado de televisión (CCTV) \cite{MIMBELA20078}. Estos métodos son bastante efectivos en el conteo de paso de vehículos \cite{CZYZEWSKI20196} y son tecnologías maduras y asentadas en el mercado, pero una de sus principales carencias es que no permiten trazar las rutas de los vehículos, a excepción de los sistemas de procesamiento de imágenes que gracias a los avances en análisis de imágenes permiten identificar los vehículos por la matrícula, pero al igual que el resto de sistemas descritos tienen un gran coste de despliegue y mantenimiento que impide realizar un despliegue masivo.\\

En los últimos años gracias al avance de las tecnologías móviles han emergido otras fuentes de datos para el análisis del tráfico basadas en la idea de recopilar la información del propio vehículo en movimiento en lugar de los datos de tráfico tradicionales obtenidos por un dispositivo fijo, esta recolección se realiza mediante los teléfonos móviles o dispositivos GPS de los conductores convirtiéndoles en una suerte de sensor en movimiento, la información enviada suele consistir en una marca de tiempo con la posición geográfica obtenida mediante GPS y/o la red móvil, este método es denominado \textit{Floating car data} (FCD), el caso más común es el de las aplicaciones de navegación GPS con las que los usuarios obtienen recomendaciones de rutas para sus trayectos basadas en las condiciones actuales del tráfico e incluso obteniendo avisos de congestión o accidentes en tiempo real, toda esta información aparentemente gratuita es ofrecida a cambio de que el usuario envíe los datos mencionados anteriormente en una especie \textit{quid pro quo}, los datos de todos los usuarios son recolectados en un sistema central que los agrega y analiza para generar la información de tráfico que estos mismos usuarios reciben. Como desventaja de este sistema cabría destacar que se basa en la voluntad del usuario para instalar la aplicación y ceder sus datos, además se da el problema de que el usuario no siempre utilice la aplicación para todos sus desplazamientos por lo que se perdería una gran cantidad de información. Otro forma de obtención de información en este tipo de sistemas se basa en los localizadores GPS de flotas de vehículos de empresas privadas como compañías de taxis o de reparto \cite{HUANG2018318} debido a la naturaleza de estas empresas los datos obtenidos suelen cubrir un gran área urbana, pero dado el perfil concreto de estos conductores es difícil extrapolar estos datos al resto de la población.\\

Los sistemas basado en FCD han mostrado ser muy efectivos y capaces de competir con los sistemas tradicionales estacionarios \cite{KESSLER2018299} \cite{NARANJO2010}  \cite{KESSLER2018299} \cite{GUILLAUME2008}, además cuenta con la ventaja de no necesitar de un despliegue de sensores aunque no hay que menospreciar el coste del desarrollo y mantenimiento de la aplicación. Estos datos no suelen ser libres y en los casos en los que sí, solo se libera parcialmente como es el caso de Google Maps \cite{LI20194}, por tanto los organismos públicos de transporte deben llegar a acuerdos con las empresas propietarias de estos datos.\\

Para suplir las desventajas expuestas anteriormente existentes propuestas basadas en el método FCD pero con un enfoque diferente respecto a la forma en la que se obtienen los datos del usuario, en lugar de hacer el usuario participe mediante el uso de una aplicación o de sus dispositivos GPS la recolección de los datos se realiza de forma inadvertida para el usuario, para ello se utilizan sensores que detectan el tráfico WiFi o Bluetooth que sus dispositivos generan continuamente, de esta forma cualquier usuario que porte un dispositivo que emita este tipo de señales contribuye a la recolección de datos.\\

En la actualidad se han desarrollado multitud de investigaciones sobre este campo, realizando análisis tanto en espacios cerrado como abiertos y con diferentes tipos de sensores, pero no solo se trata de un campo de estudio sino que ya existen casos de éxito, por ejemplo, Claude Villier \textit{et al.} en su estudio para la evaluación de estrategias de gestión de tráfico para eventos deportivos \cite{VILLIERS2019100052} utilizaron sensores Bluetooth para obtener los tiempos de viaje entre varios puntos, complementando así los datos obtenido por los detectores de bucle de inducción ya existentes, recopilaron datos durante 5 años y demostraron que es un enfoque eficaz en la recopilación de datos de tiempo de viaje para la evaluación de soluciones de gestión del tráfico. También encontramos casos de éxito en la aplicación de estas tecnologías para analizar el flujo de peatones y su comportamiento, por ejemplo,  Paul Jackson \textit{et al.} \cite{JACKSON2012} desplegaron con este propósito una red de sensores en Londres durante los Juegos Olímpicos de 2012 cuyos datos podían consultar en tiempo real desde los centros de control permitiendo a los operadores tener una mayor visibilidad sobre el comportamiento de espectadores y multitudes, ayudándoles a tomar decisiones para mitigar riesgos potenciales. \\

Por otra parte, la monitorización basada en WiFi cuenta con una ventaja sobre las basadas en Bluetooth y es que puede conllevar menor coste de despliegue ya que en algunos casos es posible utilizar infraestructuras ya existentes, por ejemplo, Martin W.Traunmuell \textit{et al.} \cite{TRAUNMUELLER20184} utilizaron los datos recolectados durante una semana por una red de WiFi pública existente con 54 puntos de acceso en el Bajo Manhattan de la ciudad de Nueva York, con estos datos realizaron un análisis espacial de redes para identificar la frecuencia y dirección de los transeúntes entre los nodos de la red, aplicándolo a la red de aceras peatonales y calles  demostrando el potencial de estas tecnologías para desarrollar modelos de movilidad eficientes en entornos urbanos densos. Otros estudios han explorado las posibilidades de este tipo de monitorización en el interior de edificios, en la investigación de Thor S.Prentow \textit{et al.} \cite{PRENTOW2015305} recolectaron datos de la red WiFi de un complejo hospitalario para estimar densidad y el flujo de personas entre diferentes áreas del hospital para ayudar a determinar si las instalaciones se utilizan de forma óptima.\\

Como se ha comentado anteriormente, la monitorización por WiFi cuenta con la ventaja de poder reutilizar infraestructuras ya existentes pero al enfocarse solo en esta tecnología inalámbrica se pierde una gran capacidad de detección aportada por los dispositivos Bluetooth como \textit{wearables} o incluso vehículos con radio equipada con Bluetooth, por esto resultan interesantes los enfoques en los que se apuesta por monitorizar ambos tipos de tráfico, por ejemplo, en la Universidad de Granada el grupo de investigación GeNeura en el transcurso de los proyectos SIPEsCa \cite{proyecto_sipesca}, PETRA \cite{proyecto_petra} y MoSos \cite{proyecto_mosos} desarrollaron un sistema distribuido de sensores basado en el ordenador de placa única de bajo coste Raspberry Pi para la detección de paquetes WiFi y Bluetooth con el objetivo de contar paso de vehículos y personas enfocado a la gestión y optimización del tráfico.\\

Los ordenadores de placa única como el Raspberry Pi están más enfocados a proyectos multipropósito en los que son necesarias más capacidades de procesamiento, por ejemplo, para la ejecución de un sistema operativo completo, además a pesar de su reducido coste comparado con sistemas convencionales sigue suponiendo una gran inversión para realizar un despliegue de grandes dimensiones como podría ser la monitorización del tráfico de una ciudad. Por estos motivos en este proyecto se plantea la idea de diseñar un sensor utilizando tecnologías  \textit{System on a chip} que permitiría reducir aún más el coste y consumo.\\

\end{document}
