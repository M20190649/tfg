\documentclass[../proyecto.tex]{subfiles}

\begin{document}

\chapter{Introducción}

El objetivo de este proyecto es desarrollar un sistema distribuido de bajo coste que permita monitorizar en tiempo real el movimiento de personas y vehículos. \\

La detección de los usuarios se realizará monitorizando el tráfico WiFi generado por los dispositivos móviles como por ejemplo los smartphones o vehículos con WiFi integrado.\\

Para realizar las detecciones se utilizará el microchips de bajo coste ESP8266 y se podrán consultar a través de una interfaz web o mediante una API desarrolladas con el microframework web Flask.\\

\section{Motivación}
En el trascurso de los proyectos SIPESCA, MOSOS y PETRA se desarrolló un sistema de detección basado en el SBC Raspberry, este dispositivo a pesar de su reducido coste comparado con sistemas convencionales sigue suponiendo una gran inversión para realizar un despliegue de grandes dimensiones como podría ser la monitorización del tráfico de una ciudad.\\

Para permitir un despliegue masivo asequible es necesario reducir el coste de los detectores y de cara a la autonomía de los detectores también surge la necesidad de disminuir el consumo energético de estos.\\






% Estado del arte
% Requisitos
% Objetivos / Propuesta
% Metodología y plan de trabajo (https://eprints.ucm.es/44402/1/tfg-denys-carlos.pdf)



https://www.crc.id.au/tracking-people-via-wifi-even-when-not-connected/

\end{document}
