\documentclass[../proyecto.tex]{subfiles}

\begin{document}

\chapter{Introducción}
\section{Motivación}

En las últimas décadas la revolución de los dispositivos móviles ha irrumpido en nuestras vidas de una forma que pocos podrían haber previsto, desde que aquel primer iPhone lanzado en 2007 iniciara el \textit{boom} de los \textit{smartphones} nuestra relación con los teléfonos móviles ha cambiado enormemente, ya no concebimos utilizar un móvil únicamente para hacer llamadas sino que se ha convertido en un centro de entretenimiento e información y en general una herramienta imprescindible en nuestro día a día.\\

Actualmente el número de usuarios únicos de móviles en el mundo se estima en 5,19 mil millones, aproximadamente el 90\% de la población adulta tiene un \textit{smartphone} y el 95\% lo usa a diario, una prueba de esto es que cada año una mayor proporción del tráfico de internet se genera desde estos dispositivos, actualmente más de la mitad del tráfico web en el mundo se realiza desde el móvil, en concreto un 53\% lo que supone un 8,6\% más que en 2019. Este crecimiento también se aprecia en los dispositivos \textit{wearables}, en 2019 el uso de estos dispositivos se cifraba en 722 millones, se estima que un 25\% de los españoles utilizan  \textit{smartwatchs} \cite{informe_ditendria}.\\

Sin duda alguna este crecimiento está ligado al gran avance de las tecnologías inalámbricas como el WiFi y el Bluetooth, gran parte de las funcionalidades de un teléfono móvil se basan en la conectividad con internet y con otros dispositivos. El avance en la fabricación de SoCs (\textit{System on a chip}) ha permitido integrar estas tecnologías inalámbricas en todo tipo de sistemas con una reducción del tamaño, coste y consumo considerables en los últimos años, permitiendo el desarrollo de dispositivos de dimensiones reducidas como \textit{smartwatch} y auriculares inalámbricos a precios asequibles.\\

Estas tecnologías inalámbricas tienen algo en común, para iniciar una conexión a un punto de acceso o una conexión \textit{ad hoc} requieren de un proceso de descubrimiento para anunciarse o reconocer a otros dispositivos o puntos de acceso, durante este proceso de forma intencionada o colateral pueden exponer su identidad públicamente ya que el envió de esta información suele realizarse en \textit{broadcast} en un medio en el cualquiera puede escuchar. Esta información a primera vista podría parecer no tener mucho valor ya que solo son datos de gestión de la conexión y nos los datos en sí a transmitir, pero nada más lejos de la realidad, si dispusiésemos un dispositivo que nos permitiera detectar estos procesos de descubrimiento en un sitio en concreto y al tratarse estos dispositivos emisores de dispositivos móviles personales podríamos obtener una estimación de la afluencia de personas en ese punto y tiempo concretos, además si tenemos en cuenta el hecho de que estos dispositivos suelen tener un identificador único podríamos incluso caracterizar a sus portadores con el fin de trazar sus rutas.\\

Una red distribuida de este tipo de sensores sería de gran utilidad al permitir recopilar información sobre los movimientos de las personas, esta información sería de gran utilidad para el análisis de movilidad urbana en grandes ciudades, con esta información los organismos podrían conocer los puntos de concentración y rutas habituales de sus ciudadanos, tanto de viandantes como conductores, lo que ayudaría en el diseño de los espacios urbanos, mejora del tráfico privado y a mejorar los sistemas de transporte público.\\

La idea de estudiar la movilidad urbana mediante este tipo de sensores plantea un gran reto ya que para poder obtener datos precisos en grandes ciudades sería necesario realizar un despliegue masivo de cientos o incluso miles de sensores, este tipo de despliegue podría suponer un alto coste desalentando a los organismos públicos a su implementación a pesar de los grandes beneficios que podría reportar a la población, por este motivo surge la motivación para realizar este proyecto en el que se desarrollará un sistema de detección de bajo coste que hará más asequibles este tipo de despliegues.\\

\section{Objetivos}

El principal objetivo de este proyecto es desarrollar un sensor de detección de presencia utilizando SoCs de bajo coste y consumo que permitan monitorizar en tiempo real el movimiento de transeúntes y vehículos mediante el análisis de tráfico WiFi y BLE generado por dispositivos móviles personales como smartphones o dispositivos \textit{wearables}, este sensor además enviará estas detecciones a un servidor central.\\

Para facilitar la visualización de estas detecciones se plantea un objetivo secundario consistente en el desarrollo de un servidor central que permita recopilar las detecciones de los sensores, este servidor constará de una API REST para facilitar el envío de las detecciones además de una interfaz web para poderlas visualizar cómodamente.\\

Cabe destacar que no es propósito de este proyecto el posterior análisis de los datos recabados por los sensores, tratándose el servidor descrito como objetivo secundario de un simple recolector centralizado de las detecciones realizadas.\\

\section{Estructura del documento}


\end{document}
